%%%%%%%%%%%%%%%%%%%%%%%%%%%%%%%%%%%%%%%%%
% University/School Laboratory Report
% LaTeX Template
% Version 3.0 (4/2/13)
%
% This template has been downloaded from:
% http://www.LaTeXTemplates.com
%
% Original author:
% Linux and Unix Users Group at Virginia Tech Wiki 
% (https://vtluug.org/wiki/Example_LaTeX_chem_lab_report)
%
% License:
% CC BY-NC-SA 3.0 (http://creativecommons.org/licenses/by-nc-sa/3.0/)
%
%%%%%%%%%%%%%%%%%%%%%%%%%%%%%%%%%%%%%%%%%

%----------------------------------------------------------------------------------------
%	PACKAGES AND DOCUMENT CONFIGURATIONS
%----------------------------------------------------------------------------------------

\documentclass{article}

\usepackage[version=3]{mhchem} % Package for chemical equation typesetting
\usepackage{siunitx} % Provides the \SI{}{} command for typesetting SI units
\usepackage{fullpage}
\usepackage{subcaption}
\setlength{\parskip}{3mm}
\usepackage[titletoc,title]{appendix}
\usepackage{xfrac}
\usepackage{tikz}
\usepackage{tablefootnote}
\usepackage{epstopdf}
\usepackage{enumitem}
\usepackage{etoolbox}
\usepackage{wrapfig}
\usepackage{longtable}
\usepackage{array}
\usepackage{setspace}
\usepackage{multirow}
\usepackage[bookmarks]{hyperref}
\usepackage{listings}
\patchcmd{\thebibliography}{\section*{\refname}}{}{}{}

\usepackage{graphicx} % Required for the inclusion of images


\usepackage{sourcecodepro}
\usepackage[default]{sourcesanspro}
\usepackage[T1]{fontenc}

% code command for including code snippets inline
% (fake verbatim, so all special character should be escaped,
% or textmode equivalents of special characters should be used)
\newcommand{\code}[1]{
    \begin{tikzpicture}[baseline=0ex] 
        \node[anchor=base, text height=1em, text depth=1ex, inner ysep=0pt, draw=black!13, fill=black!3, rounded corners=2pt] at (0,0) {
            \footnotesize\texttt{#1}
        }; 
    \end{tikzpicture}
}

% button command for including code snippets inline
% (fake verbatim, so all special character should be escaped,
% or textmode equivalents of special characters should be used)
\newcommand{\button}[1]{
    \begin{tikzpicture}[baseline=0ex] 
        \node[anchor=base, text height=1em, text depth=1ex, inner ysep=0pt, draw=black!13, fill=black!3, rounded corners=2pt] at (0,0) {
            \footnotesize{#1}
        }; 
    \end{tikzpicture}
}

\lstset{
  showspaces=false,
  showtabs=false,
  breaklines=true,
  showstringspaces=false,
  breakatwhitespace=true,
  stringstyle=\color{greenstrings},
  basicstyle=\ttfamily\color{grey}
}

%\setlength\parindent{0pt} % Removes all indentation from 

\renewcommand{\arraystretch}{1.5} % Fix the table sizes
\addtocontents{toc}{\protect\setstretch{-1}}


%\usepackage{times} % Uncomment to use the Times New Roman font

%----------------------------------------------------------------------------------------
%	DOCUMENT INFORMATION
%----------------------------------------------------------------------------------------

\title{\includegraphics[width=.3\textwidth]{/Users/ben/Dropbox/Resources/UVicLogo/UVicLogo} \\ \vspace{7mm} Milestone 3 \\ SENG 310} % Title


\author{UVic Enhancement Suite} % Author name

\date{\today} % Date for the report

\raggedright
\begin{document}

\maketitle % Insert the title, author and date

\begin{center}
\begin{tabular}{l r}
Team Members & Ben Hawker \\
 & Brendon Earl \\
 & Clair Xu \\
 & David Draker \\
 & Kent MacDonald \\
Instructor: & Erini Kalliamvakou  % Instructor/supervisor
\end{tabular}
\end{center}

% If you wish to include an abstract, uncomment the lines below
% \begin{abstract}
% Abstract text
% \end{abstract}

\tableofcontents
\pagebreak

%--------------------------------------------------------------------------------
%	PROTOTYPES
%--------------------------------------------------------------------------------

\section{Updated Prototypes}

Thanks to feedback from other teams, we found that the redirection to google maps from the UVic MySpace was less than idea. We decided to implement a popover containing the location and directions of the class, along with extra details, thereby allowing the user to maintain their workflow and recieve information quicker.

\renewcommand*{\arraystretch}{12}
\begin{longtable}[t]{p{10cm}>{\raggedright\arraybackslash}p{4cm}}
     $\vcenter{\includegraphics[width=\linewidth]{img/Home-Expanded-Event}}$ & A popover containing a map to DTB A102, a floorplan, hours of opperation, and brief notes about the building opens when the agenda item is clicked.\\
\end{longtable}

%--------------------------------------------------------------------------------
%	TESTING PLAN
%--------------------------------------------------------------------------------

\section{Testing Plan}

\subsection{Introduction}

Through user testing, we hope to determine if our website is more intuitive, and faster to use than the existing UVIC MyPage. Furthermore, we aim to identify if our site has introduced any new issues with usability.

\subsection{Method}

\subsubsection{Participants}

To answer these questions, about 20 current UVic students will participate in the study. Ideally, 10 of the participants will be from the initial-study testing pool, to perform a within-group comparison. Then to reach a total pool of 20, 10 new participants will be recruited to perform a between-groups comparison. To encourage individuals to take part in the study, all students will be offered a small chocolate for their participation. Each participant will be informed of the risks and procedures involved with the study, and will be asked to sign a consent form.

\subsubsection{Platform}

A static webpage with emulated data will be used for the study. The website can be reached via any modern web-browser with Internet access.

\subsubsection{Procedure}

Each participant will be tested in a quiet room with minimal distraction. The users will perform the study on their own computer to ensure comfort and familiarity with the system. After answering a selection of questions pertaining to relevant information (age, year of study, etc.), participants will be asked to perform a variety of tasks that will require them to navigate our site and use many of its various features. The full set of questions and tasks are located in Appendix A.

These questions were selected because they require the user to perform actions that resemble daily and yearly tasks that would typically be completed by a UVic student (checking timetable, grades, etc.).

Following the tasks, users will be asked a second set of more open-ended questions that aim to summarize the experience, and identify any further problems not already addressed.

\subsubsection{Information Collection}

Responses to the survey questions asked at the beginning and end of the study will be recorded manually by researchers.

While users are performing the tasks outlined in the middle of the study, information about mouse clicks and time spent per-page will be gathered and analyzed by analytics tools built into the website, or tabulated by hand (If we cannot get them working in time).

\subsection{Timeline}

The study will take place over five days, starting on March 21st and running until March 25th. Gathered data will be analyzed following the study and compiled by March 27th

\subsection{Responsibilities}

Each group member will be responsible for conducting four user-tests each (two participants from the initial tests and two newly recruited students).




%--------------------------------------------------------------------------------
%	Appendices
%--------------------------------------------------------------------------------
%\pagebreak
%\begin{appendices}
%
%\section{User Testing Plan}\label{ap:utesting}
%
%
%\end{appendices}


%--------------------------------------------------------------------------------
%	BIBLIOGRAPHY
%--------------------------------------------------------------------------------

%\section{References}
%
%\bibliographystyle{unsrt}
%



%--------------------------------------------------------------------------------

\end{document}