%%%%%%%%%%%%%%%%%%%%%%%%%%%%%%%%%%%%%%%%%
% University/School Laboratory Report
% LaTeX Template
% Version 3.0 (4/2/13)
%
% This template has been downloaded from:
% http://www.LaTeXTemplates.com
%
% Original author:
% Linux and Unix Users Group at Virginia Tech Wiki 
% (https://vtluug.org/wiki/Example_LaTeX_chem_lab_report)
%
% License:
% CC BY-NC-SA 3.0 (http://creativecommons.org/licenses/by-nc-sa/3.0/)
%
%%%%%%%%%%%%%%%%%%%%%%%%%%%%%%%%%%%%%%%%%

%----------------------------------------------------------------------------------------
%	PACKAGES AND DOCUMENT CONFIGURATIONS
%----------------------------------------------------------------------------------------

\documentclass{article}

\usepackage[version=3]{mhchem} % Package for chemical equation typesetting
\usepackage{siunitx} % Provides the \SI{}{} command for typesetting SI units
\usepackage{fullpage}
\usepackage{subcaption}
\setlength{\parskip}{3mm}
\usepackage[titletoc,title]{appendix}
\usepackage{xfrac}
\usepackage{tablefootnote}
\usepackage{epstopdf}
\usepackage{etoolbox}
\usepackage{listings}
\patchcmd{\thebibliography}{\section*{\refname}}{}{}{}

\usepackage{graphicx} % Required for the inclusion of images


\usepackage{sourcecodepro}
\usepackage[default]{sourcesanspro}
\usepackage[T1]{fontenc}

\newcommand{\code}{\texttt}

\lstset{
  showspaces=false,
  showtabs=false,
  breaklines=true,
  showstringspaces=false,
  breakatwhitespace=true,
  stringstyle=\color{greenstrings},
  basicstyle=\ttfamily\color{grey}
}

%\setlength\parindent{0pt} % Removes all indentation from 

\renewcommand{\arraystretch}{1.5} % Fix the table sizes


%\usepackage{times} % Uncomment to use the Times New Roman font

%----------------------------------------------------------------------------------------
%	DOCUMENT INFORMATION
%----------------------------------------------------------------------------------------

\title{\includegraphics[width=.3\textwidth]{/Users/ben/Dropbox/Resources/UVicLogo/UVicLogo} \\ \vspace{7mm} Milestone 2 \\ SENG 310} % Title


\author{UVic Enhancement Suite} % Author name

\date{\today} % Date for the report

\raggedright
\begin{document}

\maketitle % Insert the title, author and date

\begin{center}
\begin{tabular}{l r}
Team Members & Ben Hawker \\
 & Brendon Earl \\
 & Clair Xu \\
 & David Draker \\
 & Kent MacDonald \\
Instructor: & Erini Kalliamvakou  % Instructor/supervisor
\end{tabular}
\end{center}

% If you wish to include an abstract, uncomment the lines below
% \begin{abstract}
% Abstract text
% \end{abstract}

\tableofcontents

%--------------------------------------------------------------------------------
%	RESEARCH METHODS
%--------------------------------------------------------------------------------

\section{Research Methods}

To help guide our design process our team will observe users in a controlled environment to collect qualitative data on how UVic students interact with the current UVic web services, in particular, MyPage and Student Services. After some brief background questions we will ask the students to perform a number of common tasks using the UVic web services.

We believe we can gather more useful information from observing in-person interactions and collecting qualitative data than performing either a quantitative or qualitative survey. Through one-on-one user testing we hope to gain an intuitive insight about how students use UVic web services. Also, in addition to corroborating and prioritizing our current list of pain points, we hope to discover further pain points.

The process used for the user testing is listed in appendix \ref{ap:utesting}.

%--------------------------------------------------------------------------------
%	PERSONAS
%--------------------------------------------------------------------------------

\section{Personas}


%--------------------------------------------------------------------------------
%	Appendices
%--------------------------------------------------------------------------------

\begin{appendices}

\section{User Testing Plan}\label{ap:utesting}



\end{appendices}

\pagebreak

%--------------------------------------------------------------------------------
%	BIBLIOGRAPHY
%--------------------------------------------------------------------------------

%\section{References}
%
%\bibliographystyle{unsrt}
%



%--------------------------------------------------------------------------------

\end{document}